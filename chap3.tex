\chapter{À l'abordage du futur}

Si la copie est indispensable à la création et que la société cherche de plus en plus à la punir, sommes-nous condamnés à devoir courir le risque d'un procès à chaque fois que nous créerons quelque chose ?
Je pense que nous pouvons éviter cette situation, et je vais donner des pistes qui vont dans ce sens.

\section{Ne pas devenir esclave du progrès}



%(logiciel libre, donner le pouvoir aux utilisateurs, assurer le patrimoine, exemple du domaine médical)
\section{Partager pour mieux créer}
(partager le plus possible, aller contre l'aversion à la perte, exemple du téléphone, exemples de réussites du libre, exemple de Arduino, exemple de la mode)
\section{Valoriser les créateurs}
(les pirates achètent, inutilité licence globale, inventer des modèles économiques, exemple des vendeurs de glace)
\section{La réforme du système actuel}
(lutte de pouvoirs, propositions du parti pirate)
