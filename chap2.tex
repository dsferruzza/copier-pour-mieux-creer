\chapter{Interdire la copie, ou l'encourager : l'incohérence du système actuel}

Nous avons vu dans le chapitre précédent que la copie est une action naturelle que nous exerçons tous tout à long de notre vie.
Cependant, la société ne se positionne pas forcément dans ce sens, et prend une position incohérente par rapport à la copie.
Nous allons voir pourquoi dans ce chapitre.

\section{Brevet et droit d'auteur}

Nous n'allons pas aborder de manière exhaustives les notions de brevet et de droit d'auteur, car ce sont des notions complexes qui touchent beaucoup de domaines, et qui varient en fonction du pays étudié.
Je vais par contre tâcher de les définir de manière suffisante pour pouvoir les intégrer à notre réflexion sur la copie.
Sauf si je le précise explicitement, je parlerai surtout en me basant sur la législation française.

Commençons par le droit d'auteur (ou son équivalent américain, le <<~copyright~>>).
<<~Le droit d'auteur est l'ensemble des prérogatives exclusives dont dispose un auteur ou plus généralement ses ayants droit (société de production, héritiers) sur des œuvres de l'esprit originales.~>>\footnote{Source : \url{https://fr.wikipedia.org/wiki/Droit_d\%27auteur}}

Ici, on parle donc bien <<~d'œuvres de l'esprit~>>, donc immatérielles.
Bien souvent, ces idées vont être matérialisées, et placées sur des supports (papier, DVD, \dots{}) dont on ferra commerce.
Le problème, c'est que ce modèle est lucratif pour ceux qui fabriquent les objets (libraires et imprimeurs, dans l'exemple du livre) ; il n'y a finalement pas de raison spontanée première pour que ceux-ci décident de rémunérer les auteurs des œuvres qu'il exploitent.

L'idée du droit d'auteur c'est de dire que si on fait commerce de la matérialisation de l'œuvre de quelqu'un, on lui doit un pourcentage.
La démocratisation de ce mécanisme en France est souvent attribuée à Beaumarchais, qui, en temps que dramaturge, voulait toucher un pourcentage sur la recette perçu par les comédiens qui jouaient ses pièces.

On peut diviser le droit d'auteur en deux branches :
\begin{itemize}
\item \textbf{le droit moral}, qui reconnaît à l'auteur la paternité de l'œuvre et qui vise aussi le respect de l'intégrité de l'œuvre ;
\item \textbf{les droits patrimoniaux}, qui confèrent un monopole d'exploitation économique sur l'œuvre, pour une durée variable (selon les pays ou cas) au terme de laquelle l'œuvre entre dans le domaine public \textit{(nous reviendrons sur la notion de domaine public en même temps que l'explication des brevets)}.
\end{itemize}



%(historique, définition, objectifs)
\section{Interdire implique contrôler} % ou L'interdiction implique le contrôle
(DRM, HADOPI, cloud, vie privée)
\section{L'industrie du droit d'auteur}
(copyright/patent troll, rémunération du non-ajout de valeur, appropriation d'idées communes, exemple Apple)
\section{Une histoire qui se répète ?}
(exemples de "crises" similaires et résolues, exemple des bibliothèques)
\section{Partager favorise le progrès}
% sanctionner copie = sanctionner inspiration (cf chap 1)
(les pirates achètent, exemple de Arduino, exemple de la mode)
