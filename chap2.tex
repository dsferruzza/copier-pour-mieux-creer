\chapter{Interdire la copie, ou l'encourager : l'incohérence du système actuel}

Nous avons vu dans le chapitre précédent que la copie est une action naturelle que nous exerçons tous tout à long de notre vie.
Cependant, la société ne se positionne pas forcément dans ce sens, et prend une position incohérente par rapport à la copie.
Nous allons voir pourquoi dans ce chapitre.

\section{Brevet et droit d'auteur}

Nous n'allons pas aborder de manière exhaustives les notions de brevet et de droit d'auteur, car ce sont des notions complexes qui touchent beaucoup de domaines, et qui varient en fonction du pays étudié.
Je vais par contre tâcher de les définir de manière suffisante pour pouvoir les intégrer à notre réflexion sur la copie.
Sauf si je le précise explicitement, je parlerai surtout en me basant sur la législation française.

Commençons par le droit d'auteur (ou son équivalent américain, le <<~copyright~>>).
<<~Le droit d'auteur est l'ensemble des prérogatives exclusives dont dispose un auteur ou plus généralement ses ayants droit (société de production, héritiers) sur des œuvres de l'esprit originales.~>>\footnote{Source : \url{https://fr.wikipedia.org/wiki/Droit_d\%27auteur}}

Ici, on parle donc bien <<~d'œuvres de l'esprit~>>, donc immatérielles.
Bien souvent, ces idées vont être matérialisées, et placées sur des supports (papier, DVD, \dots{}) dont on ferra commerce.
Le problème, c'est que ce modèle est lucratif pour ceux qui fabriquent les objets (libraires et imprimeurs, dans l'exemple du livre) ; il n'y a finalement pas de raison spontanée première pour que ceux-ci décident de rémunérer les auteurs des œuvres qu'il exploitent.

L'idée du droit d'auteur c'est de dire que si on fait commerce de la matérialisation de l'œuvre de quelqu'un, on lui doit un pourcentage.
La démocratisation de ce mécanisme en France est souvent attribuée à Beaumarchais, qui, en temps que dramaturge, voulait toucher un pourcentage sur la recette perçu par les comédiens qui jouaient ses pièces.

On peut diviser le droit d'auteur en deux branches :
\begin{itemize}
\item \textbf{le droit moral}, qui reconnaît à l'auteur la paternité de l'œuvre et qui vise aussi le respect de l'intégrité de l'œuvre ;
\item \textbf{les droits patrimoniaux}, qui confèrent un monopole d'exploitation économique sur l'œuvre, pour une durée variable (selon les pays ou cas) au terme de laquelle l'œuvre entre dans le domaine public \textit{(nous reviendrons sur la notion de domaine public en même temps que l'explication sur les brevets)}.
\end{itemize}\bigskip

L'auteur peut jouir de ces deux droits automatiquement, dès le moment où il a créé une œuvre.
Le droit d'auteur était à la base prévu pour durer un maximum de 5 ans.
De nos jours, on parle plutôt de 70 ans après la mort de l'auteur !

On peut déjà entrapercevoir les limites d'un tel système : à partir de quand une œuvre de l'esprit, une idée, peut être appartenir à quelqu'un ?
Comme je l'ai argumenté dans le premier chapitre, il est difficile de s'approprier quelque chose d'immatériel dans le sens où cette chose n'a souvent pas de limite clairement identifiable.
D'un autre côté, ce système est une solution qui a été trouvée pour nourrir les artistes qui produisent des œuvres immatérielles.

Cependant, on constate de plus en plus de cas où le droit d'auteur est utilisé pour protéger l'artiste contre son public (parfois même à l'insu de l'artiste), chose pour laquelle il n'a pas été prévu du tout à la base.
Mais nous y reviendrons par la suite\dots{}

L'autre <<~entrave~>> à la copie est le brevet.
<<~Un brevet est un titre de propriété industrielle qui confère à son titulaire non pas un droit d'exploitation, mais un droit d'interdiction de l'exploitation par un tiers de l'invention brevetée, à partir d'une certaine date et pour une durée limitée (20 ans en général).~>>\footnote{Source : \url{https://fr.wikipedia.org/wiki/Brevet}}

Contrairement au droit d'auteur qui est automatique, il faut faire une demande pour obtenir un brevet, ce qui a un coût (très important pour un particulier).
Un brevet n'est valable que dans un État donné.
Il est cependant possible de déposer un brevet auprès de plusieurs États, mais le coût en est multiplié et sa validation n'est pas assurée partout, car les organismes de validations sont indépendants (certains organismes peuvent délivrer des brevets valables dans plusieurs États).
Les brevet sont plutôt réservés aux inventions et aux techniques industrielles.

Pour être brevetable, une invention doit correspondre à trois critères :
\begin{enumerate}
\item Elle doit être nouvelle, c'est-à-dire que rien d'identique n'a jamais été accessible à la connaissance du public, par quelque moyen que ce soit (écrit, oral, utilisation\dots{}), où que ce soit, quand que ce soit. Elle ne doit pas non plus correspondre au contenu d'un brevet qui aurait été déposé mais non encore publié.
\item Sa conception doit être inventive, c'est-à-dire qu'elle ne peut pas découler de manière évidente de l'état de la technique, pour un homme du métier.
\item Elle doit être susceptible d'une application industrielle, c'est-à-dire qu'elle peut être utilisée ou fabriquée dans tout genre d'industrie, y compris l'agriculture (ce qui exclut les œuvres d'art ou d'artisanat, par exemple).
\end{enumerate}\bigskip

Il est également exigé que la description complète de l'invention et de la manière de la reproduire doit être incluse dans le brevet, de manière à ce que le contenu technique soit disponible lors de la publication de la demande, et le reste après la fin de validité du brevet.

En effet, l'objectif premier du système de brevet est d'encourager les créateurs et inventeurs à partager leurs avancées avec le reste de la communauté.
Le problème qui se pose est que si une entreprise développe un produit et décide de le vendre, elle n'est pas à la l'abri qu'une autre entreprise récupère son produit et le vende également, mais bien moins chère car n'ayant pas de frais de recherche et développement à couvrir.
Le brevet permet de partager sa création avec la communauté, mais de conserver l'exclusivité sur son exploitation (ou sur qui peut l'exploiter) pendant une durée suffisante pour rembourser les frais de recherche et développement.
Une fois cette durée écoulée, la création tombe dans le domaine public.

On appelle <<~domaine public~>> l'ensemble des biens intellectuels qui ne sont plus protégés par le droit d'auteur ni par un brevet.

%(historique, définition, objectifs)
\section{Interdire implique contrôler} % ou L'interdiction implique le contrôle
(DRM, HADOPI, cloud, vie privée)
\section{L'industrie du droit d'auteur}
(copyright/patent troll, rémunération du non-ajout de valeur, appropriation d'idées communes, exemple Apple)
\section{Une histoire qui se répète ?}
(exemples de "crises" similaires et résolues, exemple des bibliothèques)
\section{Partager favorise le progrès}
% sanctionner copie = sanctionner inspiration (cf chap 1)
(les pirates achètent, exemple de Arduino, exemple de la mode)
