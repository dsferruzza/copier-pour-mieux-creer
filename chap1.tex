\chapter{La copie, est-ce que ça mord ?}

\begin{quote}
\textit{{\Large <<~Inventer en toute chose, c'est vouloir mourir à petit feu ; copier c'est vivre.~>>} \hspace{25pt}Honoré de Balzac -- \underline{Pierre Grassou}}
\end{quote}

\section{Qu'est-ce que la copie ?}

Le terme <<~copie~>> peut prendre plusieurs sens. Je ne vais pas parler ici de copie dans le sens <<~une feuille de copie~>>. Je parle de copie dans le sens <<~duplication~>>.

Voici les trois premières définitions du mot <<~copie~>> trouvées dans le dictionnaire\footnote{Source : Le Petit Larousse Illustré 2006} :
\begin{enumerate}
\item Reproduction exacte d'un écrit, du contenu d'un texte, d'une bande magnétique ; double, duplicata.
\item Reproduction d'une œuvre, d'un objet d'art, en principe par les mêmes techniques que celles de l'original ; réplique.
\item Imitation, calque.
\end{enumerate}\bigskip

La première chose qu'on peut remarquer, c'est que si la première définition est très claire quant au caractère <<~exact~>> de la copie, les deux définitions suivantes restent vagues sur cet aspect.
L'autre point, c'est que ces définitions n'incluent pas le sens moderne de la copie, à savoir la copie de données numériques (qui est au cœur de notre sujet).

Durant les dernières décennies, l'humanité a assisté à l'avènement de la donnée numérique : il est devenu possible, sous réserve de pouvoir l'exprimer de manière discrète\footnote{Par opposition à continue ; \url{https://fr.wikipedia.org/wiki/Système_discret}}, de stocker n'importe quelle information sur des supports numériques.
Outre leur capacité de stockage bien plus élevée que les moyens traditionnels\footnote{Pour des données discrètes}, ces supports présentent l'avantage de pouvoir être lus et écrits rapidement, et pour un coût insignifiant.
On qualifie les informations et les œuvres placées sur de tels supports <<~d'immatérielles~>> ou de <<~dématérialisées~>>, car le support (l'objet, qui est matériel donc) est dissocié des données qu'il contient.

En couplant ça avec la prolifération d'Internet, il devient possible de copier des données sans avoir à se déplacer, ni payer un coût variable important, comme le prix du papier lorsqu'on veut dupliquer un livre par exemple.
Bien sûr, il y a un coût : les infrastructures qui supportent et constituent Internet, ainsi que les terminaux des utilisateurs ont besoin d'énergie pour fonctionner.
Mais ce coût est tout de même faible par rapport aux coûts de duplication d'un livre, et peut être en grande partie considéré comme un coût fixe, c'est à dire ne dépendant pas de la quantité de données qu'on veut copier.

Par ailleurs, il se pose la question de la nuance entre copie conforme et inspiration : à partir de quand peut-on dire qu'il y a copie et non inspiration ?
Cette question est très complexe.
On peut cependant imaginer quelques réponses possibles.
On pourrait dire qu'il y a copie lorsqu'un auteur présente comme sienne une œuvre strictement identique à une œuvre existante.
Pourtant, si on copie l'ensemble d'une œuvre, mais qu'on en modifie un détail, on aurait pourtant tendance à considérer que l'œuvre obtenue a été copiée.

On pourrait, au contraire, dire qu'il y a copie lorsqu'on utilise dans une œuvre tout ou partie d'une autre œuvre, pour peu que les éléments importés n'aient pas été modifiés.
Mais on voit rapidement la limite de cette définition : y a-t-il toujours copie lorsqu'on parle d'éléments importés d'autres œuvres, mais si petits et élémentaires qu'on ne saurait identifier l'œuvre dont ils proviennent ?
Même si le résultat apparaitrait comme inspiré d'autres œuvres, il n'en resterait pas moins que l'auteur a procédé par copie pour obtenir certains éléments de son œuvre.
Le cas inverse est également envisageable : obtenir le même résultat qu'un autre, mais sans avoir eu connaissance des travaux de ce dernier.

Comme je le disais, la différence entre copier et s'inspirer n'est pas simple à exprimer.
La difficulté est de placer une limite claire et concrète, qui permet de démontrer qu'on se trouve dans un cas et pas dans l'autre.
Mais finalement, est-t-il utile de faire cette distinction ?

%(définition, différence avec inspiration)
\section{Créer implique copier ?}

Commençons par le commencement et posons-nous la question : que signifie <<~créer~>> ?
En guise de réponse, voici trois éléments d'une définition\footnote{Source : \url{https://fr.wiktionary.org/wiki/créer}} :

\begin{enumerate}
\item Tirer quelque chose du néant, faire de rien quelque chose.
\item Donner l'existence à quelque chose qui n'existait pas encore, éventuellement à partir d'autres éléments.
\item Imaginer, inventer.
\end{enumerate}\bigskip

D'après les exemples associés à ces définitions, la première serait utilisée pour désigner <<~créer~>> dans le sens religieux du terme uniquement.
Les deux autres définitions semblent être plus proches des domaines que nous étudions ici.

Il semble possible de créer à partir de rien, principalement dans deux cas : en étant un dieu, ou en imaginant.
Je ne développerai pas ici la première possibilité, car elle fait référence à un débat qui, pour le moment, nous dépasse tous.

Par contre, il semble tout à fait possible théoriquement de créer à partir de rien grâce à notre imagination.
On pourrait même dire que c'est le propre des mathématiques :

\begin{quotation}
\textit{<<~Les mathématiques se distinguent des autres sciences par un rapport particulier au réel.
Elles sont de nature purement intellectuelle, fondées sur des axiomes\footnote{Un axiome désigne une vérité indémontrable qui doit être admise} déclarés vrais (c'est-à-dire que les axiomes ne sont pas soumis à l'expérience, même s'ils en sont souvent inspirés) ou sur des postulats provisoirement admis.~>>}\footnote{Source : \url{https://fr.wikipedia.org/wiki/Mathématiques}}
\end{quotation}

Cependant, même dans le domaine des mathématiques, le réel et l'existant ont de l'importance.
En effet, de nombreux créateurs\footnote{Pas seulement mathématiciens ou même scientifiques} utilisent leur imagination pour mettre en scène des choses existantes, de manière à catalyser leur créativité.

Par exemple, Henri Poincaré, grand mathématicien français, décrit une démarche scientifique en quatre temps : préparation, incubation, illumination et vérification.
Si la première et la dernière étape relèvent beaucoup de la technique et de la méthode, les phases d'incubation et d'illumination sont en grande partie de l'ordre de l'imaginaire.
De nombreux mathématiciens\footnote{Notamment Wendelin Werner, Jacques Hadamard et Cédric Villani ; source : Sciences Humaines \no{}221} confirment que, lors de ces étapes de <<~recherche~>>, créer des images et des univers mentaux mettant en scène des idées, éléments et objets connus est une stratégie quasiment indispensable. La construction de la représentation mentale serait même indissociable de la résolution.

On peut aller encore plus loin dans l'illustration de la nécessité de s'inspirer du réel dans la démarche de création scientifique : les expériences de pensées.
Voici donc une histoire mettant en scène Galilée, célèbre savant italien du 16\ieme{} siècle.
À savoir qu'il existe d'autres histoires connues très similaires impliquant Albert Einstein ou James Clerk Maxwell, mais celle de Galilée est, à mon sens, la plus facile à réaliser soit même\footnote{Ami lecteur, je compte sur toi !}.

Tout le monde connait l'expérience de Galilée qui aurait laissé tomber deux objets du haut de la tour de Pise, l'un étant plus lourd que l'autre, pour montrer que les deux objets touchent le sol au même instant.
Cette expérience contredit la physique aristotélicienne : Galilée cherche à montrer que la vitesse de la chute des objets dans le vide est indépendante de leur masse.
Sauf qu'au début, cette expérience ne s'est jamais déroulée ailleurs que dans son imagination !

Dans sa représentation mentale de la tour de Pise, il a supposé que la masse la plus lourde allait toucher le sol avant l'autre.
Il s'est ensuite dit que, s'il attachait les deux objets ensemble, le plus léger allait ralentir le plus lourd, et donc que celui-ci toucherait le sol après un temps plus long.
Mais comme les deux masses sont attachées ensemble, on peut aussi considérer qu'elles forment une seule masse, plus importante que chacune des deux masses prises séparément.
Cette masse devrait donc, en suivant le postulat de départ, arriver au sol après un temps plus court !
Cette contradiction permet de réfuter l'hypothèse de départ, et mène à la conclusion qu'on connait.

L'expérience a, bien sûr, eu lieu de nombreuses fois par la suite ; et la loi de la chute des corps dans le vide fut confirmée.
Mais c'est imaginer un univers similaire à celui qu'il connaissait qui a permis à Galilée d'atteindre la phase <<~d'illumination~>>.
Ce type d'expériences de pensée est également très utile dans le milieu de la programmation, pour citer un autre exemple.

Là où je veux en venir, c'est qu'à part le cas, finalement très rare, voire impossible, où on crée à partir de \textbf{rien}, créer implique à un moment ou un autre s'inspirer de ce qu'on connait, qui existe déjà.
Et qui dit <<~inspiration~>> dit\dots{} copie !

C'est la thèse que défend Kirby Ferguson, un réalisateur New-Yorkais, dans son film en quatre parties : \textbf{Everything Is A Remix}\footnote{À voir absolument ! \url{http://www.everythingisaremix.info/watch-the-series/}}.

\begin{figure}[H]
\encadre{
\center
Selon lui, voici les éléments fondamentaux de la créativité :\\
\vspace{10pt}
{\LARGE Copy, Transform, Combine}\\
\textit{Copier \hspace{34pt} Transformer \hspace{34pt} Combiner}\\
\vspace{10pt}
{\Huge =}\\
\vspace{10pt}
{\LARGE Remix}
}
\caption{Les éléments fondamentaux de la créativité}
\end{figure}

Pour lui, toute création est un remix de créations existantes, qu'on a donc copiées, transformées (ou adaptées), et enfin combinées.
Il n'hésite pas à donner des exemples, dans une conférence TED\footnote{\url{http://www.ted.com/talks/kirby_ferguson_embrace_the_remix.html}}, la plupart étant issus du domaine de la musique.
Ainsi, il fait écouter des chansons de Bob Dylan et les chansons dont elles ont été <<~inspirées~>>.
On constate sans trop d'efforts que certains aspects (paroles, sons, ...) sont très similaires\dots{} voire identiques !

Un autre exemple, dans le domaine du cinéma cette fois-ci, est \textbf{Kill Bill}.
L'épopée vengeresse en deux parties de Quentin Tarantino est aujourd'hui considérée comme un des exemples les plus marqués de remix\footnote{La preuve en images : \url{http://vimeo.com/19469447}}, à tel point qu'elle en devient un hommage à toute une période cinématographique.

Les exemples ne manquent pas, parfois même de l'aveu de leurs créateurs !
Au hasard : lorsque George Lucas s'est lancé dans le projet qui deviendra par la suite \textbf{Star Wars}, il voulait à la base faire un \textit{remake} du Flash Gordon des années 30.
N'étant pas en position légale pour faire ça, il du faire évoluer son idée ; le Star Wars tel qu'on le connait comporte tout de même bon nombre de similitudes avec Flash Gordon, et rentre très bien dans la définition du remix que nous venons de voir.

L'idée que tout est un remix va bien plus loin que le domaine de la culture, et n'est pas ancrée dans notre époque. Gutenberg, connu pour avoir inventé l'imprimerie à caractères mobiles, s'est inspiré de techniques et de solutions existantes :

\begin{figure}[H]
\center
\begin{tabular}{c|c}
Presse à imprimer & 1440 apr. J.-C. \\
\hline
Caractère mobile & 1040 apr. J.-C. \\
Presse à vis & 1 apr. J.-C. \\
Encre & 180 av. J.-C. \\
Papier & 1800 av. J.-C.
\end{tabular}
\caption{Les technologies nécessaires à l'invention de l'imprimerie}
\end{figure}

Plus récent, et tout aussi connu, Henry Ford n'a pas inventé la chaine d'assemblage (1867), ni les pièces interchangeables (1801), ni l'automobile (1885).
Mais il a su combiner tous ces éléments en 1908 pour produire la première voiture en quantité industrielle : la Ford Model T.
Ainsi, il déclare :

\begin{quotation}
\textit{<<~I invented nothing new.
I simply assembled the discoveries of other men behind whom were centuries of work.
Had I worked fifty or ten or even five years before, I would have failed.
So it is with every new thing.
Progress happens when all the factors that make for it are ready, and then it is inevitable.
To teach that a comparatively few men are responsible for the greatest forward steps of manking is worst sort of nonsense.~>>}
\end{quotation}

Ce qui donne en Français :

\begin{quotation}
\textit{<<~Je n'ai rien inventé de nouveau.
J'ai simplement assemblé les découvertes d'autres hommes qui portaient derrière elles des siècles de travail.
Si j'avais travaillé cinquante, dix ou même cinq ans plus tôt, j'aurais échoué.
Il en va de même pour chaque nouvelle invention.
Le progrès se produit quand tous les facteurs qui le rendent possible sont là ; ensuite, il devient inévitable.
Dire que relativement peu d'hommes sont responsables des plus grandes avancées de l'humanité est le pire des non-sens.~>>}
\end{quotation}

Tous ces exemples montrent que la copie est une action indispensable dans le processus de création.
On a pourtant tendance à la juger de manière péjorative.
Mais nous y reviendrons\dots{}

J'aimerais terminer cette partie en te demandant, ami lecteur, comment as-tu appris à écrire ?
J'imagine que tu as commencé par recopier des lignes de lettres.
Et comment apprend-on à jouer de la musique ?
En jouant des morceaux écrits par des musiciens célèbres.
Tu sais programmer ?
Beaucoup de personnes (dont moi-même) ont commencé en copiant/collant des extraits de code sources.

\vspace{20pt}
\begin{quote}
\textit{{\Large <<~Tout lecteur devrait recopier les textes qu'il aime : rien de tel pour comprendre en quoi ils sont admirables.~>>}
\begin{flushright}Amélie Nothomb -- \underline{Le Voyage d'Hiver} (2009)\end{flushright}}
\end{quote}
\vspace{20pt}

La copie est aussi une action fondamentale pour qui veut apprendre.
Peu importe le domaine, il y a toujours une phase d'apprentissage par la copie, avant d'avoir acquis suffisamment de lucidité pour pouvoir s'en émanciper.
Plus on devient compétent dans un domaine, plus on est capable de transformer et de combiner.
Mais la copie, l'inspiration, de matériaux existants est inéluctable.

%(everything is a remix, images mentales, apprentissage, exemple de l'imprimerie, exemple de Ford)
\section{Copier n'est pas voler}

Comme je le disais précédemment, on a souvent tendance à connoter péjorativement l'action de copier.
Certains vont même jusqu'à qualifier la copie de vol : <<~si tu copies ce film au lieu de l'acheter, c'est comme si tu le volais~>>.
Copier n'est \textbf{pas} voler, et je vais expliquer pourquoi.

Voler signifie <<~s'approprier le bien d'autrui~>>\footnote{Source : \url{https://fr.wiktionary.org/wiki/voler\#Verbe.C2.A02}}.
<<~S'approprier~>> fait référence au concept de propriété.
La propriété est, d'après la Déclaration des Droits de l'Homme et du Citoyen, un <<~droit naturel et imperceptible de l'Homme~>>.
On peut la définir de la manière suivante : <<~droit d'user, de jouir et de disposer de quelque chose de façon exclusive et absolue sous les seules restrictions établies par la loi~>>\footnote{Source : Le Petit Larousse Illustré 2006}.

La propriété s'applique bien aux choses matérielles, mais peut-on posséder une idée ?
Le terme <<~propriété intellectuelle~>> est souvent employé, mais c'est un non-sens !
Il est possible d'user et de jouir d'une idée, mais on ne peut pas la détruire ni avoir des droits exclusifs dessus !
Les films ne sont pas vraiment des idées pures, mais ils sont encore moins des objets.

Bien sûr, avant qu'autrui puisse dire qu'on s'est approprié son bien, qu'on l'a volé, il faut déjà que ce bien soit effectivement sa propriété, et d'abord qu'il puisse l'être.
En ce sens, on ne peut pas vraiment dire que copier, qui est une action qui met en jeu des choses immatérielles avant tout, soit voler.

Mais ce n'est pas tout !
Je vais parler d'économie de rareté et d'économie d'abondance :

\begin{itemize}
\item L'économie de rareté, c'est l'économie des biens matériels.
Si je vends un livre que je possède, je ne l'ai plus.
\item L'économie d'abondance, c'est l'économie des biens immatériels, du savoir.
Si je vends ma prestation de professeur, une fois que j'ai fini mon cours, j'ai encore mon savoir.
\end{itemize}

Dès qu'on parle de données numériques, on est dans une économie d'abondance.
Avec une œuvre immatérielle, le coût marginal de production\footnote{Coût supplémentaire induit par la dernière unité produite} d'un exemplaire supplémentaire est nul.
L'offre est potentiellement illimitée, car Internet l'est.
La demande est limitée, car un être humain ne peut pas consommer tout le temps et a besoin de dormir.
Dans ce cas là, d'après la loi de l'offre et de la demande, le prix de vente se doit d'être nul !

On ne peut pas appliquer les lois de l'économie de rareté à l'économie d'abondance ; ça ne marche pas !
Si on pouvait voler une œuvre immatérielle (considérant alors celle-ci selon le modèle de la rareté), alors on aurait volé une œuvre qui a un prix nul.

De plus, comme j'ai commencé à le dire en définissant l'économie de rareté, voler c'est soustraire.
Si je vole le livre d'autrui, autrui n'a plus son livre.
Copier, que ça soit pour des données numériques ou même de manière générale, c'est multiplier.
Si je copie le livre d'autrui, autrui a toujours son livre.

Thomas Jefferson, résume bien le caractère naturel de la copie\footnote{Source : \url{http://reformedroitauteur.sploing.fr/\#htoc30}} :

\begin{quotation}
\textit{<<~Si la nature a rendu une chose moins susceptible que les autres de propriété exclusive, c'est l'action de penser appelée idée, qu'un individu peut posséder pour autant qu'il la garde pour lui-même, mais qui ne lui appartient plus dès qu'elle est divulguée aux autres, sans que les autres puissent s'en débarrasser.
Ce trait tout à fait particulier des idées fait que l'on ne les possède pas moins si les autres les possèdent, parce que tous les possèdent entièrement.
Instruire quelqu'un ne diminue pas mon instruction.
Il est illuminé sans me faire pour autant de l'ombre.
Que les idées devraient circuler librement d'un point à l'autre du globe, pour l'instruction morale mutuelle des hommes et pour l'amélioration de leur condition, semble avoir été volontairement conçu par la nature quand elle nous a rendu\dots{} incapable de les enfermer ou de nous les approprier.~>>}
\end{quotation}

Je conclurais par une citation de Jérémie Zimmerman, porte parole et co-fondateur de la Quadrature du Net\footnote{\url{http://www.laquadrature.net/fr}} :

\begin{quote}
\textit{{\Large <<~Quand on vous dit que pirater c'est voler, on essaie de vous faire croire que la multiplication c'est la soustraction.~>>}}
\end{quote}

Il n'hésitera pas à faire la comparaison avec 1984 de Georges Orwell : <<~la guerre c'est la paix, l'ignorance c'est la force, l'esclavage c'est la liberté~>>.
Nous verrons dans le chapitre suivant s'il est légitime de faire un tel parallèle avec le monde totalitaire de Big Brother.

%(rareté vs abondance, copier = multiplier, propriété)
