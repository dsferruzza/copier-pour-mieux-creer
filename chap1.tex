\chapter{La copie, est-ce que ça mord ?}

\begin{quote}
\textit{{\Large <<~Inventer en toute chose, c'est vouloir mourir à petit feu ; copier c'est vivre.~>>} \hspace{25pt}Honoré de Balzac -- \underline{Pierre Grassou}}
\end{quote}

\section{Qu'est-ce que la copie ?}

Le terme <<~copie~>> peut prendre plusieurs sens. Je ne vais pas parler ici de copie dans le sens <<~une feuille de copie~>>. Je parle de copie dans le sens <<~duplication~>>.

Voici les trois premières définitions du mot <<~copie~>> trouvées dans le dictionnaire\footnote{\textbf{NOM DU DICO}} :
\begin{enumerate}
\item Reproduction exacte d'un écrit, du contenu d'un texte, d'une bande magnétique ; double, duplicata.
\item Reproduction d'une œuvre, d'un objet d'art, en principe par les mêmes techniques que celles de l'original ; réplique.
\item Imitation, calque.
\end{enumerate}\bigskip

La première chose qu'on peut remarquer, c'est que si la première définition est très claire quant au caractère <<~exact~>> de la copie, les deux définitions suivantes restent vagues sur cet aspect. L'autre point, c'est que ces définitions n'incluent pas le sens moderne de la copie, à savoir la copie de données numériques (qui est au cœur de notre sujet).

(définition, différence avec inspiration)
\section{Créer implique copier ?}
(everything is a remix, images mentales, apprentissage, exemple de l'imprimerie, exemple de Ford)
\section{Copier n'est pas voler}
(rareté vs abondance, copier = multiplier, propriété)
