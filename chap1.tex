\chapter{La copie, est-ce que ça mord ?}

\begin{quote}
\textit{{\Large <<~Inventer en toute chose, c'est vouloir mourir à petit feu ; copier c'est vivre.~>>} \hspace{25pt}Honoré de Balzac -- \underline{Pierre Grassou}}
\end{quote}

\section{Qu'est-ce que la copie ?}

Le terme <<~copie~>> peut prendre plusieurs sens. Je ne vais pas parler ici de copie dans le sens <<~une feuille de copie~>>. Je parle de copie dans le sens <<~duplication~>>.

Voici les trois premières définitions du mot <<~copie~>> trouvées dans le dictionnaire\footnote{\textbf{NOM DU DICO}} :
\begin{enumerate}
\item Reproduction exacte d'un écrit, du contenu d'un texte, d'une bande magnétique ; double, duplicata.
\item Reproduction d'une œuvre, d'un objet d'art, en principe par les mêmes techniques que celles de l'original ; réplique.
\item Imitation, calque.
\end{enumerate}\bigskip

La première chose qu'on peut remarquer, c'est que si la première définition est très claire quant au caractère <<~exact~>> de la copie, les deux définitions suivantes restent vagues sur cet aspect.
L'autre point, c'est que ces définitions n'incluent pas le sens moderne de la copie, à savoir la copie de données numériques (qui est au cœur de notre sujet).

Durant les dernières décennies, l'humanité a assisté à l'avènement de la donnée numérique : il est devenu possible, sous réserve de pouvoir l'exprimer de manière discrète, de stocker n'importe quelle information sur des supports numériques.
Outre leur capacité de stockage bien plus élevée que les moyens traditionnels, ces supports présentent l'avantage de pouvoir être lus et écrits rapidement, et pour un coût insignifiant.
On qualifie les informations et les œuvres placées sur de tels supports <<~d'immatérielles~>> ou de <<~dématérialisées~>>, car le support (l'objet, qui est matériel donc) est dissocié des données qu'il contient.

En couplant ça avec la prolifération d'Internet, il devient possible de copier des données sans avoir à se déplacer, ni payer un coût variable important, comme le prix du papier lorsque l'on veut dupliquer un livre par exemple.
Bien sûr, il y a un coût : les infrastructures qui supportent et constituent Internet, ainsi que les terminaux des utilisateurs ont besoin d'énergie pour fonctionner.
Mais ce coût est tout de même faible par rapport aux coûts de duplication d'un livre, et peut être en grande partie considéré comme un coût fixe, c'est à dire ne dépendant pas de la quantité de données qu'on veut copier.

Par ailleurs, il se pose la question de la nuance entre copie conforme et inspiration : à partir de quand peut-on dire qu'il y a copie et non inspiration ?
Cette question est très complexe.
On peut cependant imaginer quelques réponses possibles.
On pourrait dire qu'il y a copie lorsqu'un auteur présente comme sienne une œuvre strictement identique à une œuvre existante.
Pourtant, si on copie l'ensemble d'une œuvre, mais qu'on en modifie un détail, on considérera pourtant 	que l'œuvre obtenue a été copiée.

On pourrait, au contraire, dire qu'il y a copie lorsqu'on utilise dans une œuvre tout ou partie d'une autre œuvre, pour peu que les éléments importés n'aient pas été modifiés.
Mais on voit rapidement la limite de cette définition : y a-t-il toujours copie lorsqu'on parle d'éléments importés d'autres œuvres, mais si petits et élémentaires qu'on ne saurait identifier l'œuvre dont ils proviennent ?
Même si le résultat apparaitrait comme inspiré d'autres œuvres, il n'en resterait pas moins que l'auteur a procédé par copie pour obtenir certains éléments de son œuvre.
Le cas inverse est également envisageable : obtenir le même résultat qu'un autre, mais sans avoir eu connaissance des travaux de ce dernier.

Comme je le disais, la différence entre copier et s'inspirer n'est pas simple à exprimer. La difficulté est de placer une limite claire et concrète, qui permet de démontrer qu'on se trouve dans un cas et pas dans l'autre. Mais avant ça, faire une telle distinction entre la copie de l'inspiration a-t-elle du sens ?

%(définition, différence avec inspiration)
\section{Créer implique copier ?}
(everything is a remix, images mentales, apprentissage, exemple de l'imprimerie, exemple de Ford)
\section{Copier n'est pas voler}
(rareté vs abondance, copier = multiplier, propriété)
