\chapter*{Introduction}
\addcontentsline{toc}{chapter}{Introduction} 

\begin{quote}
{\Large \textit{<<~Le piratage, c'est du vol !~>>}}
\end{quote}

C'est ce que l'on entend de nos jours lorsqu'on va voir un film au cinéma.
Avertissements du FBI sur nos DVD, condamnations des <<~pirates~>> à de fortes amendes, protections anti-copie : la culture se porte mal.
C'est le discours que les médias nous tiennent, et c'est ce qui est en train de devenir une vérité pour la plupart d'entre nous.
% Manque d'exemple ?

Pourtant, les chiffres tiennent un tout autre discours : l'industrie de la culture engrange chaque année davantage de bénéfices.
On voit sortir dans nos salles de cinéma des \textit{blockbusters} de plus en plus gigantesques en terme de budget et de nombre d'entrées.

Ce qui est sûr, c'est que ces dernières décennies ont été riches en terme d'innovations technologiques, notamment dans les moyens de communication, avec la prolifération des ordinateurs et d'Internet.
Il est maintenant très facile d'échanger des informations, de les copier, et de les stocker.
L'éloignement géographique est de moins en moins un problème, et chaque créateur peut espérer une visibilité à la hauteur de son talent.

Mais alors qu'en est-il vraiment ?
La copie incarnée par Internet va-t-elle détruire la créativité et dénaturer la culture tout en mettant au chômage ceux qui l'alimentent ?
Ou alors est-ce l'invention majeure de notre temps : celle qui réduira les inégalités et répandra le savoir à travers le monde ?

Passionné de nouvelles technologies, de cinéma et de jeux vidéo, j'observe depuis plusieurs années ce problème se muer en combat, et peut-être bientôt en guerre.
Aujourd'hui, j'ai décidé d'essayer de tirer cette histoire au clair.
Je suis convaincu que nous sommes à l'aube de changements majeurs dans nos sociétés, et il est de notre devoir de citoyens de nous assurer que ces changements convergent bien vers l'obtention d'un monde \textbf{meilleur} pour \textbf{tous}.

Mon objectif ici est d'apporter une vision globale de la situation, des choses qui sont en jeu, et des moyens d'action dont nous disposons.

Mon discours s'organisera en trois étapes.
J'aborderai tout d'abord les notions de base qui entourent et rassemblent l'action de copier et celle de créer.
Ces mécanismes sont importants en tant que socle et centre du problème en même temps.
Je décrirai ensuite la position actuelle de la société vis à vis de la copie, et j'insisterai sur l'incohérence de cette position.
À partir de là, il sera plus facile de comprendre les enjeux de la situation, pour aborder la troisième partie : les pistes pour un monde meilleur.
