\chapter*{Introduction}
\addcontentsline{toc}{chapter}{Introduction} 

\begin{quote}
{\Large \textit{<<~Le piratage, c'est du vol !~>>}}
\end{quote}

C'est ce qu'on entend de nos jours lorsque l'on va voir un film au cinéma. Avertissements du FBI sur nos DVD, condamnations des <<~pirates~>> à de fortes amendes, protections anti-copie : la culture se porte mal. C'est le discours que les médias nous tiennent, et c'est ce qui est en train de devenir une vérité pour la plupart d'entre nous.

Pourtant, les chiffres tiennent un tout autre discours : l'industrie de la culture engrange chaque année plus de bénéfices. On voit sortir dans nos salles de cinéma des \textit{blockbusters} de plus en plus gigantesques en terme d'argent investi et de nombre d'entrées.

Ce qui est sûr, c'est que ces dernières décennies ont été riches en terme d'innovation technologiques, notamment dans les moyens de communication, avec la prolifération des ordinateurs et d'internet. Il est maintenant très facile d'échanger des informations, de les copier, et de les stocker. L'éloignement géographique est de moins en moins un problème, et chacun peut espérer une visibilité à la hauteur de son talent.

Mais alors qu'en est-il ? La copie incarnée par internet va-t-elle détruire la culture et mettre au chômage tout ceux qui l'alimentent ? Ou alors est-ce l'invention majeure de notre temps : celle qui réduira les inégalités et répandra le savoir à travers le monde ?

Étant passionné de nouvelles technologies, mais aussi de cinéma et de jeux vidéo, cela fait quelques années que j'observe de loin ce combat monter en puissance. Aujourd'hui, j'ai décidé de me renseigner et de tirer cette histoire au clair. Je suis convaincu que nous sommes à l'aube de changements majeurs dans nos sociétés, et il est de notre devoir de citoyens de nous assurer que ces changements convergent vers l'obtention d'un monde \textbf{meilleur} pour \textbf{tous}.

Mon objectif ici est d'apporter une vision globale de la situation, des choses qui sont en jeu, et des moyens d'actions dont nous disposons.

Mon discours s'organisera en trois étapes. Je reviendrai sur les notions de bases qui entourent et rassemblent l'action de créer et celle de copier. C'est important car ce sont des mécanismes qui sont à la base de tout et au centre de notre problème. Je décrirai ensuite la position actuelle de la société vis à vis de la copie, mais surtout l'incohérence de cette position. À partir de là, il sera plus facile de comprendre les enjeux de la situation, pour aborder la 3\ieme{} partie : les pistes pour un monde meilleur.
