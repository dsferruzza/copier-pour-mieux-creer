\chapter*{Conclusion}
\addcontentsline{toc}{chapter}{Conclusion}

Nous avons vu qu'il n'est pas vraiment possible de dissocier les actions de créer et de copier.
La copie est quelque chose que nous faisons tous les jours, y compris lorsque nous apprenons.
Nous sommes nous-mêmes littéralement le résultat d'une copie : celle de nos cellules.
Les idées, et autres œuvres de l'esprit, ne fonctionnent pas comme de simples objets, et ne peuvent pas vraiment être compatibles avec la notion de propriété.

Cependant, la société actuelle continue de maintenir les systèmes du brevet et du droit d'auteur.
Ces systèmes n'ont eu que peu de légitimité depuis leur création, et sont aujourd'hui complètement décalés par rapport à une réalité d'œuvres de plus en plus numériques.
La copie ne devrait pas être interdite, mais devrait au contraire être encouragée dans une démarche gagnant- gagnant.
Sinon, aller jusqu'au bout de son interdiction reviendrait à instaurer un état totalitaire, où l'espionnage des citoyens est fait de manière massive.

Heureusement, le futur s'éclaircit sur ce point.
La philosophie du libre est peut être l'un des moyens de rédemption du système actuel.
Elle permet de ne pas devenir esclave du progrès et de nos propres machines, en confiant le pouvoir aux utilisateurs et non pas aux créateurs.
Cet état d'esprit a montré au fil des années qu'il était source d'une grande et saine créativité (il suffit de voir, par exemple, le nombre d'inventions ingénieuses réalisées par des enfants et basées sur des cartes Arduino\footnote{Carte électronique programmable et <<~open hardware~>>\\Voir \url{http://arduino.cc/fr/Main/DebuterIntroduction}}).

Mon opinion est qu'il est important de soutenir les logiciels libres.
Bien évidemment, il est possible pour ça de donner bien des choses : ses compétences, son temps, son implication, des ressources \dots{}

Mais il est aussi possible de faire bien plus simple : de les utiliser et d'en parler.
Il existe en effet d'excellents logiciels libres, que j'utilise personnellement tous les jours.
On peut citer \href{https://www.mozilla.org/fr/firefox/new/}{Mozilla Firefox}, \href{https://www.mozilla.org/fr/thunderbird/}{Mozilla Thunderbird}, \href{https://www.videolan.org/vlc/}{VLC}, \href{http://fr.libreoffice.org/}{LibreOffice}, \href{http://www.gimp.org/}{GIMP} et même certains contenus en ligne majeurs comme \href{https://fr.wikipedia.org/}{Wikipédia} ou \href{http://www.openstreetmap.org/}{OpenStreetMap}.
Plus ces logiciels seront utilisés, plus ils gagneront en crédibilité et pourront nous protéger contre leurs alternatives propriétaires qui nous délestent progressivement de nos droits fondamentaux.

La solution se trouve aussi du côté du Parti pirate.
Peu m'importe si celui-ci arrive au pouvoir ou pas, tant que les idées qu'il véhicule sont mises en œuvre.
Pour l'heure, le moyen le plus simple est de le faire monter en puissance, et de le soutenir.
J'y penserai sérieusement si j'ai un jour l'occasion de voter pour un de ses représentants.
Et j'espère que ce jour arrivera bientôt.

Ami lecteur, si toi aussi tu veux un monde meilleur, tu sais ce qu'il te reste à faire.

% soutenir le libre